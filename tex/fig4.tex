\documentclass[11pt]{article}


\usepackage{amssymb,amsfonts}
\usepackage[figuresright]{rotating}
\usepackage{amsmath,amssymb}
\usepackage{graphicx}% Include figure files
\usepackage{dcolumn}% Align table columns on decimal point
\usepackage{bm}% bold math
\usepackage{amscd,amsthm}
\usepackage{ifthen}
\usepackage{mathtools}

\usepackage{pgf,tikz}
\usepackage{pgfplots}
\usetikzlibrary{calc,shadows}
\usetikzlibrary{pgfplots.groupplots}
\usepackage{subfigure}
\usepackage{url}
\usetikzlibrary{pgfplots.groupplots}

%\usetikzlibrary{external}
%\tikzexternalize[prefix=./]
%\tikzset{external/force remake}

\renewcommand{\S}{S}
\renewcommand{\L}{{L}} 
\newcommand{\N}{{N}} 
\newcommand{\TL}[1]{\tilde{#1}} 
\newcommand{\vy}{\mathbf y}
\newcommand{\vn}{\mathbf n}
\newcommand{\complexset}{\mathbb C}



\usetikzlibrary{pgfplots.groupplots}
\usetikzlibrary{matrix,arrows,decorations.pathmorphing}
\usepackage{pgfplots,pgfplotstable}
\usepgfplotslibrary{fillbetween}
\pgfplotsset{compat=1.10}
\usetikzlibrary{matrix,arrows,decorations.pathmorphing}

\newcommand{\errorband}[7]{ \pgfplotstableread{#1}\datatable \addplot
  [name path=pluserror,draw=none,no markers,forget plot] table
  [x={#2},y expr=\thisrow{#3}+\thisrow{#4}] {\datatable};

  \addplot [name path=minuserror,draw=none,forget plot]
    table [x={#2},y expr=\thisrow{#3}-\thisrow{#4}] {\datatable};

  \addplot [forget plot,fill=#5,opacity=#6]
    fill between[on layer={},of=pluserror and minuserror];

  \addplot [#5,#7,very thick]
    table [x={#2},y={#3}] {\datatable};
}

\begin{document}

\begin{figure}[h!]
\begin{center}
\begin{tikzpicture}[scale=1.1]
\begin{groupplot}[
         title style={at={(0.5,-0.25)},anchor=north}, group
         style={group size=2 by 1, horizontal sep=1.5cm },
         width=0.45\textwidth, xticklabel style={/pgf/number
           format/fixed, /pgf/number format/precision=3}, yticklabel
         style={/pgf/number format/fixed, /pgf/number
           format/precision=5 }, ymin=0,xmin = 0.65, xmax = 0.99,
         extra x ticks = {0.65,0.99}, extra x tick labels =
         {0.65,0.99}, ymode=log, ] % left panel 
%
\nextgroupplot[xlabel
  = {$M_{\max}$},ymax=1400000,ylabel = {sample complexity},title={(a)}]
    \errorband{../fig/comparison_vary_closeness_linsep.dat}{1}{3}{4}{blue}{0.2}{solid}
    \addlegendentry{AR} %
    \errorband{../fig/comparison_vary_closeness_linsep.dat}{1}{6}{7}{red}{0.2}{densely dotted}
    \addlegendentry{PLPAC} %
    \errorband{../fig/comparison_vary_closeness_linsep.dat}{1}{9}{10}{brown}{0.2}{dashed}
    \addlegendentry{BTMB} % 
\addplot [black,draw,very thick,dashdotted] table[x index=1,y
  index=2]{../fig/comparison_vary_closeness_linsep.dat};
\addlegendentry{$H(\tau(M^{(\eta)}))$}

               
% right panel
\nextgroupplot[ymax=1200000,xlabel = {$M_{\max}$},
ylabel = {}, title={(b)}]
  \errorband{../fig/comparison_vary_closeness_extreme.dat}{1}{3}{4}{blue}{0.2}{solid}
  \addlegendentry{AR} %
  \errorband{../fig/comparison_vary_closeness_extreme.dat}{1}{6}{7}{red}{0.2}{densely dotted}
  \addlegendentry{PLPAC} %
  \errorband{../fig/comparison_vary_closeness_extreme.dat}{1}{9}{10}{brown}{0.2}{dashed}
  \addlegendentry{BTMB}
       \addplot [black,draw,very thick,dashdotted] table[x index=1,y
         index=2]{../fig/comparison_vary_closeness_extreme.dat};
       \addlegendentry{$H(\tau(M^{(\xi)}))$}
\end{groupplot}          
\end{tikzpicture}
\end{center}
%\caption{\label{fig:btlcomparision} (a) Empirical sample complexity of
%  the AR, PLPAC, and BTMB algorithms applied to the BTL model
%  $\pmat^{(\eta)}$ with parameters $\parw_i = \log(\eta + \numitems -
%  i), i = 1,\ldots,10$, and (b) applied to the BTL model
%  $\pmat^{(\xi)}$ with parameters $\parw_i = \xi (\numitems-i),
%  i=1,\ldots,10$, as a function of $\pmatmax \defeq \max_{i,j}
%  \pmat_{ij}$. For panel (a) and (b) we varied $\eta$ and $\xi$ such
%  that $\pmatmax \in [0.65,0.99]$.  The error bars correspond to one
%  standard deviation from the mean.  While the AR algorithm has even
%  lower sample complexity than the PLPAC and BTMB algorithms in the
%  regime where $\pmatmax$ is not to close to $1$; the PLPAC and BTMB
%  perform better when $\pmatmax$ is close to one.}
\end{figure}

\end{document}
